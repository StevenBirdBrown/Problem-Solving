\documentclass[twocolumn, 10.5pt]{article}
\usepackage{verbatim}
\usepackage{amsfonts}
\usepackage{geometry}
\usepackage{amsmath}
\usepackage{amsthm}
\usepackage{amssymb}
\usepackage{listings}
\usepackage{graphicx}
\usepackage{clrscode3e}
\usepackage{txfonts}
\usepackage{fontspec}
%\usepackage{ctex}
\usepackage{float}
\usepackage{enumerate}
\setmainfont{Times New Roman}
\geometry{top=2.5cm,bottom=2.5cm,left=2.5cm,right=2.5cm}
\setlength\parindent{0em}
\begin{document}
	\title{Problem Solving Homework (Week 7)}\author{161180162 Xu Zhiming}\maketitle
	\section*{JH Chapter 4}
	\subsection*{4.2.1.4}
	\begin{proof}
		Suppose that the cost obtained by such algorithm is $cost(GMS(I))$, and the optimal cost is $Opt_{MS}(I)$, then we know that:
		\[
		\begin{aligned}
			cost(GMS(I))-Opt_{MS}(I)\le p_k\le Opt_{MS}(I)\\
			\therefore \frac{cost(GMS(I))-Opt_{MS}(I)}{Opt_{MS}(I)}\le 1
		\end{aligned}
		\]
		Consequently, \proc{Graham's Algorithm} is a 2-approximation algorithm for \proc{MS}, too.
	\end{proof}
	\subsection*{4.2.1.5}
	One possible input instance can be:
	\[
	\begin{pmatrix}
		p_1\\
		p_2\\
		p_3\\
		p_4\\
		\vdots\\
		p_{r-4}\\
		p_{r-3}\\
		p_{r-2}\\
		p_{r-1}\\
		p_r
	\end{pmatrix}=
	\begin{pmatrix}
		2n-1\\
		2n-1\\
		2n-2\\
		2n-2\\
		\vdots\\
		n+1\\
		n+1\\
		n\\
		n\\
		n
	\end{pmatrix}
	\]
	These $2n+1$ tasks make $R_{GMS}=3/4$, which is largest.
	\subsection*{4.2.3.3}
	\begin{proof}
		\begin{enumerate}[(i)]
			\item $dist(G,c)$\par
			Since in $L_\delta$, $c(\{u,p\})+c(\{p,v\})>c(\{u,v\})$, then:
			\[
			\frac{c(\{u,v\})}{c(\{u,p\})+c(\{p,v\})}<1,\ \forall u,v,p\text{ allowed}
			\]
			\[
			\max\left\{	\frac{c(\{u,v\})}{c(\{u,p\})+c(\{p,v\})}-1\right\}<0
			\]
			Therefore,
			\[
			\max\left\{0,\max\left\{	\frac{c(\{u,v\})}{c(\{u,p\})+c(\{p,v\})}-1\right\}\right\}=0
			\]
			That means that $h_L:\ L\rightarrow \mathbb{R}^{\ge 0}$, and $h_L$ satisfies the first property.\par 
			As for the second one, the time complexity for computing $h_L$ is bounded by $$O\left(\binom{|V(G)|}{2}\cdot (|V(G)|-2)\right)=O(|V(G)|^3)$$
			Thus, it is polynomial-time computable.		
			\item $dist_k(G,c)$\par 
			Alike what's shown above, the value of the inner maximal function is negative. Hence the value of the outer maximal function is 0. Besides, the time complexity is bounded by $$O\left(k\cdot \binom{|V(G)|}{2}\right)=O(|V(G)|^2)$$
			\item The range of parameter $k$ only makes the time complexity increase to $O(|V(G)|^3)$, which is still polynomial.
		\end{enumerate}
	\end{proof}
	\subsection*{4.2.3.4}
	\begin{proof}
		\begin{enumerate}[a)]
			\item The first property of distance function is ensured by (i). Now consider the second one. Since $h_{index}(u)$ equals the order of $u$, it can be computed in polynomial time. Therefore, $h_{index}$ is a distance function of $\overline{U}$ according to $L_I$.
			\item 
		\end{enumerate}
	\end{proof}
	\subsection*{4.2.3.5}
\end{document}