\documentclass[twocolumn]{article}
\usepackage{verbatim}
\usepackage{amsfonts}
\usepackage{geometry}
\usepackage{amsmath}
\usepackage{amsthm}
\usepackage{amssymb}
\usepackage{listings}
\usepackage{ctex}
\usepackage{graphicx}
\usepackage{clrscode3e}
\usepackage{txfonts}
\usepackage{fontspec}
\usepackage{float}
\usepackage{enumerate}
\setmainfont{Times New Roman}
\geometry{top=2.5cm,bottom=2.5cm,left=2.5cm,right=2.5cm}
\setlength\parindent{0em}
\begin{document}
	\title{Problem Solving Homework (Week 13)}\author{161180162 Xu Zhiming}\maketitle
	\section*{JH Chapter 5}
	\subsection*{5.3.2.5}
	使用下面5.3.3.10中的算法
	\subsection*{5.3.3.2}
	2, 4, 5, 8, 10, 11, 13, 16, 17, 19, 20, 22, 23, 25, 26, 29, 31, 32, 34, 37, 38, 40, 41, 43, 44, 46, 47, 50, 52, 53, 55, 58, 59, 61, 62.\par 
	The code for computing these values is listed below:
	\begin{lstlisting}[language=C]
#include <cstdio>
#include <cmath>
#include <set>
using namespace std;
inline int resq(int a, int b, int p)
{
    int c = a, d = 1;
    for(int i=0;b;++i)
    {
        if(b&0x1)
        {
            d = d * c % p;
            c = c * c % p;
        }
        b>>=1;
    }
    return d;
}
int n = 63;
int main()
{
    set<int> s;
    for(int i=1;i<n;++i)
    {
        int tmp = resq(i, (n-1) / 2, n);
        if(tmp!=1&&tmp!=-1)
        {
            for(int j=1;j<n;++j)
            {
                if(i*j%n==1)
                {
                    s.insert(i);
                    s.insert(j);
                }
            }
        }
    }
    set<int>::iterator iter;
    for(iter = s.begin(); iter!=s.end(); ++iter)
    printf("%d ", *iter);
    return 0;
} 
	\end{lstlisting}
	\subsection*{5.3.3.9}
		All the numbers below satisfy \textbf{Definition 5.3.3.7}.
	\begin{proof}
		\begin{enumerate}[(i)]
			\item \[
			\begin{aligned}
				\because a^2\cdot b^2\mod p&=a^2\mod p\cdot b^2\mod p\\
				\therefore Leg\left[\frac{a\cdot b}{p}\right]&=Leg\left[\frac{a}{p}\right]\cdot Leg\left[\frac{b}{p}\right]\\
				\therefore Jac\left[\frac{a\cdot b}{p}\right]&=Jac\left[\frac{a}{p}\right]\cdot Jac\left[\frac{b}{p}\right]
			\end{aligned}
			\]
			\item 
			\[
			\begin{aligned}
				\because a&\equiv b\mod n\\
				\therefore a^2&\equiv b^2\mod n\\
				\therefore Jac\left[\frac{a}{p}\right]&=Jac\left[\frac{b}{p}\right]
			\end{aligned}
			\]
			\item
				Suppose $a=q_1^{j_1}\cdot q_2^{j_2}\cdot\cdots\cdot q_m^{j_m}$,$p_i$ is prime and $j_i$ is positive interger. 
			\[
			\begin{aligned}
			Jac\left[\frac{n}{a}\right]&=\prod_{i=1}^{m}\left(Leg\left[\frac{n}{q_i}\right]\right)^{j_i}\\
			&=\prod_{i=1}^{m}\left(n^{(q_i-1)/2}\mod q_i\right)^{j_i}
			\end{aligned}
			\] 
			\item
			\[
			\begin{aligned}
				\because 1^2=1&\mod p\\
				\therefore Leg\left[\frac{1}{n}\right]&=1\\
				\therefore Jac\left[\frac{1}{n}\right]&=1
			\end{aligned}
			\]
			\item 
			\[
			\begin{aligned}
				\because Jac\left[\frac{2}{n}\right]&=\prod_{i=1}^{l}\left(Leg\left[\frac{a}{p_i}\right]\right)^{k_i},\ n=p_1^{k_1}p_2^{k_2}\cdot\cdots\cdot p_l^{k_l}\\
				n&=8k+3,8k+5,Jac\left[\frac{2}{n}\right]=-Jac\left[\frac{n}{2}\right]\text{(iii)}\\
				&=-Jac\left[\frac{1}{2}\right]\text{(ii)}\\
				&=-1\\
				n&=8k+1,8k+7,Jac\left[\frac{2}{n}\right]=-Jac\left[\frac{n}{2}\right]\text{(iii)}\\
				&=-Jac\left[\frac{1}{2}\right]\text{(ii)}\\
				&=-1
			\end{aligned}
			\]
		\end{enumerate}
	\end{proof}
	\subsection*{5.3.3.10}
	\begin{proof}
		Since $a$ and $n$ are coprimes, we can first use property (iii), reducing to calculate $(-1)^{\frac{a-1}{2}\cdot\frac{n-1}{2}}\cdot Jac\left[\frac{n}{a}\right]$. The power of $-1$ can be easily computed in $O(1)$ times since we only need to determine whether the exponential is even or odd. $Jac\left[\frac{n}{a}\right]=Jac\left[\frac{n\mod a}{a}\right]$ according to property (ii). This time, again, use property (iii), reducing to compute $Jac\left[\frac{a}{n\mod a}\right]$, which is similar to what \proc{Eulerian-Algorithm} does for calculating $\gcd(a,n)$, with time complexity in $O(\log n)$. In the end, $Jac\left[\frac{a}{n}\right]$ can be reduced in property (v)'s form, where only some powers of $-1/1$ are computed in $O(1)$. Therefore, for every odd $n$ and every $a\in\left\{1,2,\cdots,n-1\right\}$ with $\gcd(a,n)=1$, $Jac\left[\frac{a}{n}\right]$ can be computed in polynomial time according to $\log_2n$.
	\end{proof}
\end{document}