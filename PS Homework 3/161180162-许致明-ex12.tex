\documentclass[twocolumn]{article}
\usepackage{verbatim}
\usepackage{amsfonts}
\usepackage{geometry}
\usepackage{amsmath}
\usepackage{amsthm}
\usepackage{amssymb}
\usepackage{listings}
\usepackage{graphicx}
\usepackage{clrscode3e}
\usepackage{txfonts}
\usepackage{fontspec}
\usepackage{float}
\usepackage{enumerate}
\setmainfont{Times New Roman}
\geometry{top=2.5cm,bottom=2.5cm,left=2.5cm,right=2.5cm}
\setlength\parindent{0em}
\begin{document}
	\title{Problem Solving Homework (Week 12)}\author{161180162 Xu Zhiming}\maketitle
	\section*{JH Chapter 5}
	\subsection*{5.2.2.7}
	\begin{enumerate}[(i)]
		\item First, in order to achieve realize the random choice of prime $p$, $c\lceil\log_2n\rceil$ bits are needed. Besides $s=\proc{Number}(x)\mod p$ requires another $c\lceil\log_2n\rceil$ bits. Both $p$ and $s$ are sent, therefore, the communication complexity is $2c\lceil\log_2n\rceil$.
		\item Suppose $x\neq y$ while $$\proc{Number}(x)\mod p=\proc{Number}(y)\mod p$$
		$$\therefore \left(h=\left|\proc{Number}(x)-\proc{Number}(y)\right|\right)\equiv 0\mod p$$
		Since $x,y\in{0,1}^n$, $h$ is less than $2^n$, i.e., $h$ has fewer than $n$ different prime divisors, which means that at most $n-1$
		primes $l_i\in\left\{2,3,\dots,n^c\right\}$ have the property
		$$\proc{Number}(x)\mod l_i=\proc{Number}(y)\mod l_i$$
		Therefore, the probability that $R_1$ randomly chooses a prime with the property mentioned above for the given input $(x,y)$ is at most
		$$\frac{n-1}{n^c/\ln n^c}\le \frac{c\ln n}{n^{c-1}}$$
		$$\therefore Prob\left(\left(R_1,R_2\right)\text{ accept }(x,y)\right)\ge 1-\frac{c\ln n}{n^{c-1}}$$
 	\end{enumerate}
	\subsection*{5.2.2.8}
	\begin{enumerate}[(i)]
		\item 
		\begin{proof}
			Suppose we use deterministic algorithm to compute $Equality_n$, then we need to compare each pair $\{x_i,y_i\},i=1,2,\dots,n$. This process needs at least $n$ communication complexity since at least $n$ bits from either $x$ or $y$ should be transmitted to another computer for comparison.
		\end{proof}
		\item 
		Suppose $C_1$ and $C_2$ share enough random $0-1$ strings, say, $O(n)$ ones. Then $C_1$ randomly picks one string, and sends its index (the length of which is $O(\log_2n)$) to $C_2$. This is a two-sided-error algorithm and
		\[
		\begin{aligned}
			P(Equality_n(x,y)=1)\ge\frac{2}{3},\ \text{if $x=y$}\\
			P(Equality_n(x,y)=0)\ge\frac{2}{3},\ \text{if $x\neq y$}
		\end{aligned}
		\]
		\item 
		\begin{proof}
			Since one-sided-error algorithm accepts every input $(x,y)$ only if $x= y$. Then to make this happens, we need to verify every single bit pair $(x_i,y_i)$ for $i=1,2,\dots,n$. Otherwise, leave out some bit pairs unchecked will cause our algorithm accept $(x,y)$ while $x\neq y$. In order to do this, alike what's mentioned in (i), at least $n$ bits should be transmitted. Therefore, the lower bound of communication complexity for one-sided-error algorithm with regard to $Equality_n$ is $n$.
		\end{proof}
	\end{enumerate}
\end{document}