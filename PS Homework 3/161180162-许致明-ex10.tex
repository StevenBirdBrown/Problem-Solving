\documentclass[twocolumn, 10.5pt]{article}
\usepackage{verbatim}
\usepackage{amsfonts}
\usepackage{geometry}
\usepackage{amsmath}
\usepackage{amsthm}
\usepackage{amssymb}
\usepackage{listings}
\usepackage{graphicx}
\usepackage{clrscode3e}
\usepackage{txfonts}
\usepackage{fontspec}
%\usepackage{ctex}
\usepackage{float}
\usepackage{enumerate}
\setmainfont{Times New Roman}
\geometry{top=2.5cm,bottom=2.5cm,left=2.5cm,right=2.5cm}
\setlength\parindent{0em}
\begin{document}
	\title{Problem Solving Homework (Week 10)}\author{161180162 Xu Zhiming}\maketitle
	\section*{Min-Weight Perfect Matching}
	\begin{codebox}
		\li Start with a dual solution\li 
		\textbf{until} perfect matching found in subgraph of tight edges\zi  \textbf{do}\li 
		\If tight edges have no perfect matching\Then\li 
		find Hall set and modify dual values accordingly, \zi 
		expanding the  subgraph\End
	\end{codebox}
	\section*{JH Chapter 4}
	\subsection*{4.3.5.6}
	\begin{enumerate}[(a)]
		\item 
			\begin{enumerate}[Step 1:]
				\item Construct a minimal spanning tree $T$ of $G$ according to $c$.
				\item Perform depth-first-search of $T$ from $q$, and order the vertices such that $s$ is the last one. Let $H$ be the resulting sequence.
			\end{enumerate}
			\begin{enumerate}[Output:]
				\item The path $H$.
			\end{enumerate}
		\item 
			\begin{enumerate}[Step 1:]
				\item Construct a minimal spanning tree $T$ of $G$ according to $c$.
				\item $S:=\left\{v\in V|\deg_T(v)\text{ is odd}\right\}$.
				\item Compute a minimum-weight perfect matching $M$ on $S$ in $G$.
				\item Create the multigraph $G'=\left(V,E(T)\cup M\right)$ and construct an Eulerian tour $\omega$ starting from $q$, ending on $s$ in $G'$.
				\item Construct the path $p$ by removing all repetitions of the occurrences of every vertex in $\omega$
			\end{enumerate}
			\begin{enumerate}[Output:]
				\item $p$
			\end{enumerate}
	\end{enumerate}
	\subsection*{4.3.5.11}
	\begin{proof}
		First, we prove that it is polynomial. Since Step 1 constructs a minimal spanning tree, this can be done in $O(V^2)$ time. Step 2 performs a depth-first-search, which is of $O(V+E)$. Therefore, the total time complexity is bounded by $O(V^2)$. Hence, it is a polynomial-time algorithm.	\par 
		Second, we prove it is $2\cdot (1+r)$-approximation for $$(\Sigma_I,\Sigma_O,L,Ball_{r,distance}(L_\Delta),\mathcal{M},cost,minimum)$$
		By modifying the proof of \emph{Theorem 4.3.5.2}, we can easily obtain the proof for this problem.
		\[
		\begin{aligned}
			cost(T)&\le \sum_{c\in E(T)}c(e)\le cost(H_{Opt})\\
			cost(W)&=2\cdot cost(T)\\
			cost(W)&\le 2\cdot cost(H_{Opt})\\
		\end{aligned}
		\]
		\[
		\begin{aligned}
			&\text{The (1+r)-triangle inequality tells us:}\\
			cost(\overline{H})&\le (1+r)\cdot cost(W)\\
			\therefore cost(\overline{H})&\le cost(W)\le 2\cdot(1+r)\cdot cost(H_{Opt})
		\end{aligned}
		\]
	\end{proof}
	With regard to the instance presented in \emph{Fig. 4.15}, Algorithm 4.3.5.1 is better than \proc{Christofides Algorithm}.
	\subsection*{4.3.5.13}
	Alike what's shown for \proc{Christofides Algorithm}, it is
	$(r,O(n^{\log_2 ((1+r)^2)}))$-quasistable for $dist$.
\end{document}