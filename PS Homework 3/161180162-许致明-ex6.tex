\documentclass[twocolumn, 10.5pt]{article}
\usepackage{verbatim}
\usepackage{amsfonts}
\usepackage{geometry}
\usepackage{amsmath}
\usepackage{amsthm}
\usepackage{amssymb}
\usepackage{listings}
\usepackage{graphicx}
\usepackage{clrscode3e}
\usepackage{txfonts}
\usepackage{fontspec}
%\usepackage{ctex}
\usepackage{float}
\usepackage{enumerate}
\setmainfont{Times New Roman}
\geometry{top=2.5cm,bottom=2.5cm,left=2.5cm,right=2.5cm}
\setlength\parindent{0em}
\begin{document}
	\title{Problem Solving Homework (Week 6)}\author{161180162 Xu Zhiming}\maketitle
	\section*{JH Chapter 3}
	\subsection*{3.7.2.1}
	\begin{enumerate}[(i)]
		\item 
		minimize
		\[
		\sum_{i=1}^{n}c_ix_i
		\]
		under the constraints
		\[
		\begin{aligned}
			\sum_{i=1}^{n}a_{ji}x_i=b_j&\text{ for $j\in M$}\\
			\sum_{i=1}^{n}a_{ri}x_i+s_r=b_r&\text{ for $r\in\{1,\dots,m\}-M$}\\
			x_i\ge 0&\text{ for $i\in Q$}\\
			x_i+q_i\ge 0&\text{ for $i\in \{1,\dots,n\}-Q$}\\
			s_i\ge 0&\text{ $\forall s_i$}
		\end{aligned}
		\]
		\item 
		minimize
		\[
		\sum_{i=1}^{n}c_ix_i
		\]
		under the constrains
		\[
		\begin{aligned}
			\sum_{i=1}^{n}a_{ji}x_i\ge b_j+s_j&\text{ for $j\in M$}\\
			\sum_{i=1}^{n}a_{ri}x_i\ge b_r&\text{ for $r\in\{1,\dots,m\}-M$}\\
				x_i\ge 0&\text{ for $i\in Q$}\\
			x_i+q_i\ge 0&\text{ for $i\in \{1,\dots,n\}-Q$}\\
			s_i\ge 0&\text{ $\forall s_i$}
		\end{aligned}	
		\]
		\item 
		minimize
		\[
		c^T\cdots X
		\]
		under the constraints
		\[
		\begin{aligned}
			\sum_{i=1}^{n}a_{ji}x_i\ge b_j&\text{ for $j\in M$}\\
			\sum_{i=1}^{n}-a_{ji}x_i\ge -b_j&\text{ for $j\in M$}\\
			\sum_{i=1}^{n}a_{ri}x_i\ge b_r&\text{ for $r\in \left\{1,\dots,m\right\}-M$}\\
			x_i\ge 0&\text{ for $i\in Q$}
		\end{aligned}
		\]
	\end{enumerate}
	\subsection*{3.7.2.4}
	maximize
	\[
	\sum_{e\in E}x_e
	\]
	under the $|V|$ constrains
	\[
	\sum_{e\in E}x_e=1 \text{for every $v\in V$}
	\]
	and the following $|E|$ constrains
	\[
	x_e\in \{0,1\} \text{for every $e\in E$}
	\]
	\subsection*{3.7.2.5}
	Let $x_{ij}$ be 1 if edge $ij$ is in the tree $T$. Then we know that:
	\begin{itemize}
		\item Exactly $n-1$ edges are in $T$;
		\item There is no cycle in $T$.
	\end{itemize}
	Suppose the weight function is $w$, i.e., $w(i,j)$ denotes the weight of edge $ij$. The formulation is:\par 
	minimize
	\[
	\sum_{ij\in T.E}w_{ij}
	\]
	under the first constraint
	\[
	\sum_{ij\in E}x_{ij}=n-1
	\]
	and the second constraint
	\[
	\sum_{ij\in E;i\in S,j\in S}x_{ij}\le |S|-1,\ \forall S\subseteq V
	\]
	%4/12/16
	\subsection*{3.7.4.4}
	\begin{proof}
		For step 1-3 in Algorithm 3.7.4.2, we changed the objective function to weighted sum $\sum_{i=1}^{n}c_i\alpha_i$. The inequality $\sum_{h\in Index(a_j)}\alpha_h\ge 1$ still holds. Therefore, $\exists t\in Index(a_j)$, where $\alpha_r\ge 1/k$. The rounding is that $\beta_i=1\Leftrightarrow a_i\ge 1/k,i=1,\dots,m$. Consequently, $\beta_i\le k\alpha_i$, and $\sum_{i=1}^{n}c_i\beta_i\le k\sum_{i=1}^{n}c_i\alpha_i$.\par \
		Based on what is stated above, the assertion still holds.
	\end{proof}
	\subsection*{3.7.4.12}
	\begin{enumerate}[(i)]
		\item   It can be proved by the following procedure. First, all feasible solutions in LP consists of the convex hull. Second, the optimal solution must be on the vertexes of it, and the boolean solutions make up the vertexes. In the end, non-boolean solutions must be on the lines (except vertexes). Therefore, $Opt_{LP}(I(G))=Opt_{MMP}(G)$.
		\item 	I'm sorry that I can't solve this problem.
	\end{enumerate}
	\subsection*{3.7.4.16}
	The linear relaxation for \proc{SCP$(k)$} is:\par 
	minimize
	\[
	\sum_{i=1}^{m}x_i
	\]
	under the constraints
	\[
	\begin{aligned}
	\sum_{h\in Index(a_j)}x_h\ge 1&\text{ for $j=1,\dots,n$}\\
	x_i\ge 0&\text{ for $i=1,\dots, m$}	
	\end{aligned}
	\]
	The dual problem is:\par 
	maximize
	\[
	\sum_{i=1}^{n}y_i
	\]
	under the constraints
	\[
	\begin{aligned}
	\sum_{j\in S_i}y_j\le 1&\text{ for $i=1,\dots,m$}\\
	y_i\ge 0&\text{ for $i=1,\dots,n$}
	\end{aligned}
	\]
	With the primal-dual scheme, we can solve the above problem. Denote their solutions by $\alpha=(\alpha_1,\dots,\alpha_n)$ and $\beta=(\beta_1,\dots,\beta_m)$, respectively. Then we have $\beta_i=1\Leftrightarrow \sum_{j\in S_i}\alpha_j=1$.
\end{document}