\documentclass{article}
\usepackage{geometry}
\usepackage{amsmath}
\usepackage{amssymb}
\usepackage{verbatim}
\usepackage{amsthm}
\usepackage{amsfonts}
\usepackage{listings}
\geometry{top=2.0cm,bottom=2.0cm,left=2.0cm,right=2.0cm}
\begin{document}
\title{Problem Solving Homework(Week 13)}\author{161180162 Xu Zhiming}\date{\today}\maketitle
\setlength\parindent{0em}
\large\textbf{TC}\\
\textbf{16.1-2}
In this algorithm, every step we make a choice and get a subproblem. So it's a greedy algorithm.
Write the pseudocode of it.
\begin{lstlisting}[language=C,numbers=left]
GREEDY-ACTIVITY-SELECTIOR-REVISED(s,f)
//Prelimenary: the activities are sorted using start time
n=s.length
k=n
for i from n-1 down to 1
	if f[i]<=f[k]
		A.push(a[i])	
		k=i
return A
\end{lstlisting}
The proof of the correctness:
\begin{proof}
Consider any nonempty subproblem $S^k$, and let $a_m$ be an activity in $S_k$ with the last start time.
Next I will prove that $a_m$ is included in some maximum-size of subset of mutually compatible activities
of $S_k$. Let $A_k$ be a maximum-size subset of mutually compatible activities in $S_k$, and let $a_j$ be 
the activity in $A_k$ with the last start time. If $a_i=a_j$, it's done. If $a_i\neq a_j$, let the set 
$A_k^{'}=A_k-\{a_j\}\cup\{a_m\}$ be $A_k$ but substituting $a_m$ for $a_j$. The activities in $A^k_{'}$ are
disjoint, which follows because the activities in $A-k$ are disjoint, $a_j$ is the last activity in $A_k$ to
start and $s_m>=s_j$. Since $|A_k^{'}|=|A_k|$, we conclude that $A_k^{'}$ is a maximum=size subset of mutually
compatible activities of $S_k$, and it includes $a_m$.
\end{proof}
\textbf{16.1-3}\\
Example for case 1:\\
s 1 4  7 11\\ 
f 6 8 10 15\\
The set S in textbook can be an example for case 2 and 3\\ 
\textbf{16.2-1}\\
We need to prove local greedy choices will eventually lead to a global optimal solution.
\begin{proof}
In this problem, since the material we take can be fractional. That's to say, we can take as much as we want, partial
or whole. Then we can assign a unit price which is price divided by weight to it. It's obvious that we shall take the
material of highest unit price as much as possible. If not, we choose another material of a lower unit price, for that
weight it takes, it's not optimal(Since we could have obtained a larger total price). So the local optimal choice will
lead to a global optimal choice, the greedy choice property holds.  
\end{proof}
\textbf{16.2-2}
\begin{lstlisting}[language=C,numbers=left]
KNAPSACK(w[1..n],v[1..n],limit)
let DP[0..limit] be a new array
for i from 0 to limit
	DP[i]=0
for i from 1 to i
	for j from limit down to w[i]
		DP[j]=max(DP[j],DP[j-w[i]]+v[j])
return DP[limit]
\end{lstlisting}
\textbf{16.3-2}
\begin{proof}
Obviously, if a node has only one child, then it can't be a keyword but an ordinary node.
Then we can delete it and move all its decedents up 1 level, shortening the code length of
them by 1. Therefore, a tree that not full can't correspond to an optimal prefix tree.
\end{proof}
\textbf{16.3-5}
\begin{proof}
Assum we have k elements, their frequencies are $f_1\le f_2\le\dots\le f_k$, and one kind of optimal 
prefix code makes them stay in depth $d_1,d_2,\dots,d_k$. Then we have $W(k)=\sum_{i=1}^{k}f_i\times d_i$
is a minimum. Suppose the depths are monotonically increasing, $W^{'}(k)=\sum_{i=1}^{k}f_i\times d_i$, where
$d_1\ge d_2\ge\dots\ge d_k$. According to sequence inequality, $W^{'}(k)\le W(k)$. Thus we are done with
the proof.
\end{proof}
\textbf{16.3-8}
\begin{proof}
	Alike the proof of Lemma 16.2, 
\begin{equation*}
\begin{aligned}
B(T)-B(T')&=\sum_{c\in C}c.freq\cdot d_T(c)-\sum_{c\in C}c.freq\cdot d_{T'}(c)\\
&=(c.freq-c.freq)(d_T(c)-d_{T'}(c))\\
&=0
\end{aligned}
\end{equation*}
Therefore, Huffman code isn't more efficient than ordinary code.
\end{proof}
\textbf{16-1}\\
\textbf{a.}
\begin{lstlisting}[language=C,numbers=left]
COIN-CHANGE(n)
let q,d,n,p all be 0	//Initializing,numbers of every kind of coins
remain=n
while(remain>0)
	if(remain>15)
		q++
		remain-=15
	else if(remain>10)
		d++
		remain-=10
	else if(remain>5)
		n++
		remain-=5
	else
		p+=remain
		remain=0
return q,d,n,p
\end{lstlisting}
\begin{proof}
Optimal Substructure: In an optimal solution, for a certain amount of money, 
suppose we need at least n coins. If there is another way to change the money and requires less coins, 
then we can substitute this way for the former ways and yield a better solution, which contradicts 
our assumption. Therefore, the optimal substructure is satisfied.\\
Greedy Choice: If we keep choosing coins with the highest price since the remaining amount permits, 
it's obvious that we can minimize the total number of coins used.
\end{proof}
\textbf{b.}
\begin{proof}
	We can view the available coin denominations $c^0,c^1,\dots,c^k$ as the radixes of c-radix numbers.  
	It's trivial that any number can be denoted by the radix. So we can use these coins to change any amount
	of money. Alike the proof in problem a, this problem also has the two properties. So it can be solved by
	a greedy algorithm.
\end{proof}
\textbf{c.}\\
1 cent, 8 cents and 10 cents. 17 cents' case.\\ 
\textbf{d.}
\begin{lstlisting}[language=C,numbers=left]
COIN-CHANGE(d[1..k])	//d is the denominations of coins
for(i = 1; i <= n; i++)
	for(j = 0; d[j] <= i&&j< = k; j++)
		if DP[i] == 0 or DP[i-d[j]]+1 < DP[i]
			DP[i] = DP[i-d[j]]+1
\end{lstlisting}
\end{document}
