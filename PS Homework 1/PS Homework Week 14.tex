\documentclass{article}
\usepackage{verbatim}
\usepackage{geometry}
\usepackage{amsmath}
\usepackage{amsthm}
\usepackage{amssymb}
\usepackage{listings}
\usepackage{amsfonts}
\geometry{top=2.0cm,bottom=2.0cm,left=2.0cm,right=2.0cm}
\begin{document}
\title{Problem Solving Homework(Week 14)}\author{161180162 Xu Zhiming}\date{\today}\maketitle
\setlength\parindent{0em}
\large\textbf{TC}\\
\normalsize\textbf{21.1-2}
\begin{proof}
\textbf{Sufficiency:} If two vertices(u and v) are in the same component, then in line 4 and 5 in CONNECTED-COMPONENTS(G), they are put in the same set by method UNION(u,v).\\
\textbf{Necessity:} In CONNECTED-COMPONENT(G), method UNION is only used in line 5, when u and v are connected by an edge, that's to say, u and v are in the same set if u and v are in the same component.
\end{proof}
\textbf{21.1-3}\\
Assume that at first we have $|V|$ components and they are not connected. At last, we have only $k$ components. Performing UNION(G) once can reduce the number of components by 1. So the total number of performing UNION(G) is $|V|-k$. As for FIND-SET(u,v), it is performed $2\times|E|$\\ times, because for each edge $uv$, FIND-SET(u,v) will be performed twice.\\
\textbf{21.2-1}
\begin{lstlisting}[language=C,numbers=left]
//Linked-list and weighted union
MAKE-SET(x[i])
let S[i] be a new linked-list
S[i].head=x[i]
S[i].tail=x[i]
x[i].next=NIL
x[i].p=S[i]
S[i].length=1
FIND-SET(x)
return x.p.head
UNION(x,y)
if x.p.length>y.p.length
	tmp_x=x.p.tail
	tmp_y=y.p.head
	do
		tmp_x.next=tmp_y
		tmp_y.p=tmp_x.p
		tmp_x.p.tail=tmp_y
		tmp_x=tmp_y
		tmp_y=tmp_y.next
	when
		tmp_y!=NIL
	x.length+=y.length
else
	tmp_y=y.p.tail
	tmp_x=x.p.head
	do
		tmp_y.next=tmp_x
		tmp_x.p=tmp_y.p
		tmp_y.p.tail=tmp_x
		tmp_y=tmp_x
		tmp_x=tmp_x.next
	when
			tmp_x!=NIL
	y.length+=x.length
\end{lstlisting}
\textbf{21.2-3}\\
For MAKE-SET and FIND-SET, $m$ times performances require $O(m)$ time individually. So the amortized time bound is $O(1)$.\\
For UNION, every time it is performed, the size of a set increases to at least twice as large as before. So, union $n$ one-element set requires at most $O(nlgn)$ times modifications to the elements in the smaller set. So each UNION running amortized running time is $O(lgn)$.\\
\textbf{21.2-6}\\
Let head has two member pointers: "first" and "second", "first" pointing to a list, "second" points to the other.\\ 
\textbf{21.3-1}\\
\textbf{21.3-2}
\begin{lstlisting}[language=C,numbers=left]
NONRECURSIVE-FIND-SET(x)
xx=x
while xx!=xx.p
	xx=xx.p
while(x!=x.p)
	tmp=x.p
	x.p=xx
	x=tmp
return x.p
\end{lstlisting}
\textbf{21.3-3}
\begin{lstlisting}[language=Python,numbers=left]
for i=1 to n
	MAKE-SET(x[i])	//n
for i=1 to n
	UNION(x[i],x[1])//n
for i=1 to m-(2*n-1)
	FIND-SET(x[n])  //(m-2n+1)lgn
\end{lstlisting}
Overall: $\Omega(2n+(m-2n+1)lgn)=\Omega(mlgn)$\\
\textbf{21-1}\\
\textbf{a:}
$extrace[6]=4,3,2,6,8,1$.\\
\textbf{b:}
\begin{proof}
\textbf{Loop invariant:} Every time before the for-loop starts, the set that contains i has the index that indicates i's position in array $extracted$.\\
\textbf{Initialize:} When $i=1$, i is the minimum element globally, so the set that contains 1 will eject it first.\\
\textbf{Maintenance:} When the $k^{th}$ loop is over, the ${k+1}^{th}$ loop starts. Assume that $k\in K_s$, then $K_r=K_s\cup K_{r}$($K_r$ exits and r is the smallest number larger than s). Than if $k+1$ originally belongs to $K_r$, than it will be ejected from it.    
\end{proof}
\textbf{c:}
\begin{lstlisting}[language=C,numbers=left]
OFF-LINE-MINIMUM(m,n)	//Disjoint-set data structure
for i=1 to n
	if	FIND-SET(i) belongs to K[j]
		if	j!=m+1
			extracted[j]=i
			let l be the smallest value greater than j for which set K[l] exits
			UNION(K[l],K[j])
return extracted
\end{lstlisting}
\end{document}