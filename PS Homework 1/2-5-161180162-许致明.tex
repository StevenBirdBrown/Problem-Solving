\documentclass[twocolumn]{article}
\usepackage{verbatim}
\usepackage{amsfonts}
\usepackage{geometry}
\usepackage{amsmath}
\usepackage{amsthm}
\usepackage{amssymb}
\usepackage{listings}
\usepackage{graphicx}
\usepackage{clrscode3e}
\usepackage{txfonts}
\usepackage{enumerate}
\usepackage{ctex}
\usepackage{txfonts}
\usepackage{fontspec-xetex}
\usepackage{float}
\geometry{top=2.5cm,bottom=2.5cm,left=2.5cm,right=2.5cm}
\setlength\parindent{0em}
\setmainfont{Times New Roman}
\begin{document}
	\title{问题求解(二)作业(第五周)}\author{161180162 许致明}\maketitle
	\section*{CS第四章}
	\subsection*{4.1-16}
	使用相同方法分解后,假设不再成立,归纳法不能证明结论。
	\subsection*{4.1-17}
	$$n_\text{顶点}=n_\text{三角}+2$$
	假定对于$n$边形成立,则将$n$边形相隔一个点的两个顶点相连,得到了一个$n-1$边形,它的顶点数少1,且三角形数也少1,依此推至三角形仍然成立。故上述关系成立。
	\subsection*{4.2-8}
	\[
	\begin{aligned}
		T(0)&=2000\\
		T(n)&=2\cdot T(n-1)+2000,\ n\ge 2\\
		\therefore T(n)&=1000\times (2^{n+1}-2)
	\end{aligned}
	\]
	\subsection*{4.2-11}
	\[
	\begin{aligned}
		T(n)&=2T(n-1)+n\cdot 2^n,\ T(0)=1\\
		\rightarrow \frac{T(n)}{2^n}&=\frac{T(n-1)}{2^{n-1}}+n\\
		\rightarrow a_n&=\frac{T(n)}{2^n}\\
		\rightarrow a_n-a_{n-1}&=n\\
		\therefore a_n&=\frac{n(n+1)}{2}+a_0\\
		&=\frac{n(n+1)}{2}+1\\
		\therefore T(n)&=2^n\cdot a_n\\
		&=2^n+{2^{n-1}n(n+1)}
	\end{aligned}
	\]
	\subsection*{4.3-9}
	\begin{enumerate}[a.]
		\item 
			\begin{table}[H]
			\begin{tabular}{ll|ll}
				\textbf{Number}&\textbf{Size}&\textbf{Work}&\textbf{Work total}\\ \hline
				1&$n$&$n$&$n$\\ 
				8&$n/2$&$n/2$&$4n$\\
				&$\cdots$&$\cdots$&\\
				$8^{\log n}=n^3$&1&$1$&$1\cdot n^3=n^3$
			\end{tabular}
		\end{table}
		\[
		\begin{aligned}
			T(n)&=\sum_{i=0}^{\log n}4^in\\
			&=n\sum_{i=0}^{\log n}4^i\\
			&=n\frac{1-4^{(\log n)+1}}{1-4}\\
			&=n\left(\frac{4n^2-1}{3}\right)\\
			&=\Theta(n^3)
		\end{aligned}
		\]
		\item 
		\begin{table}[H]
			\begin{tabular}{ll|ll}
				\textbf{Number}&\textbf{Size}&\textbf{Work}&\textbf{Work total}\\ \hline
				1&$n$&$n^3$&$n^3$\\ 
				8&$n/2$&$(n/2)^3$&$n^3$\\
				&$\cdots$&$\cdots$&\\
				$8^{\log n}=n^3$&1&$1$&$1\cdot n^3=n^3$
			\end{tabular}
		\end{table}
		\[
		\begin{aligned}
		T(n)&=\sum_{i=0}^{\log n}n^3\\
		&=n^3\sum_{i=0}^{\log n}1\\
		&=n^3\cdot \log n\\
		&=\Theta(n^3\log n)
		\end{aligned}
		\]
		\item 
			\begin{table}[H]
			\begin{tabular}{ll|ll}
				\textbf{Number}&\textbf{Size}&\textbf{Work}&\textbf{Work total}\\ \hline
				1&$n$&$n$&$n$\\ 
				3&$n/2$&$n/2$&$3n/2$\\
				&$\cdots$&$\cdots$&\\
				$3^{\log n}$&1&$1$&$1\cdot 3^{\log n}$
			\end{tabular}
		\end{table}
		\[
		\begin{aligned}
		T(n)&=\sum_{i=0}^{\log n}\left(\frac{3}{2}\right)^in\\
		&=n\sum_{i=0}^{\log n}\left(\frac{3}{2}\right)^i\\
		&=n\cdot\left(\frac{1-\left(\frac{3}{2}\right)^{(\log n)+1}}{1-\frac{3}{2}}\right)\\
		&=\Theta(3^{\log n})
		\end{aligned}
		\]
		\item 
			\begin{table}[H]
			\begin{tabular}{ll|ll}
				\textbf{Number}&\textbf{Size}&\textbf{Work}&\textbf{Work total}\\ \hline
				1&$n$&$1$&$1$\\ 
				1&$n/4$&$1$&$1$\\
				&$\cdots$&$\cdots$&\\
				$1$&1&$1$&$1$
			\end{tabular}
		\end{table}
		\[
		\begin{aligned}
		T(n)&=\sum_{i=0}^{\log n}1\\
		&=\Theta(\log n)
		\end{aligned}
		\]
		\item 
		\begin{table}[H]
			\begin{tabular}{ll|ll}
				\textbf{Number}&\textbf{Size}&\textbf{Work}&\textbf{Work total}\\ \hline
				1&$n$&$n$&$n$\\ 
				3&$n/3$&$n/3$&$n$\\
				&$\cdots$&$\cdots$&\\
				$3^{\log n}=n$&1&$1$&$1\cdot n=n$
			\end{tabular}
		\end{table}
		\[
		\begin{aligned}
		T(n)&=\sum_{i=0}^{\log n}n\\
		&=\Theta(n\log n)
		\end{aligned}
		\]
	\end{enumerate}
	\subsection*{4.4-1}
	\begin{enumerate}[a.]
		\item \emph{Case c: }$T(n)=\Theta(n^3)$
		\item \emph{Case b: }$T(n)=\Theta(n^3\log n)$
		\item \emph{Case c: }$T(n)=\Theta(n^{\log_2 3})$ 
		\item \emph{Case b: }$T(n)=\Theta(\log n)$
		\item \emph{Case a: }$T(n)=\Theta(n^2)$
	\end{enumerate}
	\subsection*{4.4-6}
	\begin{proof}
		第$i$层总花费:
		$$a^i\left(\frac{n}{b^i}\right)^c=n^c\left(\frac{a}{b^c}\right)^i$$
		最底层花费:
		$$\log_b n\times d$$
		则:$$T(n)=n^c\sum_{i=0}^{\log_b n-1}\left(\frac{a}{b^c}\right)^i+\log_b n\times d$$
		可得:
		\begin{enumerate}[1.]
			\item $\log_b a<c$时,$a/b^c<1$,$T(n)=\Theta(n^c)$
				\item $\log_b a=c$时,$a/b^c=1$,$T(n)=\Theta(n^c\log n)$
					\item $\log_b a>c$时,$a/b^c>1$,$T(n)=\Theta(n^{\log_b a})$
		\end{enumerate}
	\end{proof}
	\subsection*{4.5-8}
	\begin{proof}
		对于$n=1$的基本情况,$T(1)=O(1)=d$成立;\\
		假定对于$n=2^k$,$T(n)=n^3$成立,下证对于$n=2^{k+1}$,此解$T(n)=n^3$仍成立:
		\[
		\begin{aligned}
			T(2^{k+1})&=8T(2^k)+2^k\log 2^k\\
			&=O(8\cdot(2^k)^3)+2^k\log 2^k\\
			&=O((2^{k+1})^3)+2^k\log 2^k\\
			&=O((2^{k+1})^3)\\
			\therefore T(n)&=O(n^3)
		\end{aligned}
		\]
	\end{proof}
	\subsection*{4.5-9}
	\begin{proof}
		对于$n=1$的基本情况,$T(1)=d\ge c$成立\\
		当$n=2$时,$T(2)=8T(1)+2=8d+2\ge 8\cdot 8$\\
		假设$$T\left(\frac{n}{2}\right)\ge c\left(\frac{n}{2}\right)^3=c\cdot \frac{n^3}{8}$$
		则
		\[
			\begin{aligned}
				T(n)&\ge 8\cdot c\frac{n^3}{8}=\frac{cn^3}{8}\\
				\therefore T(n)&=\Omega(n^3)\\
				\therefore T(n)&=\Theta(n^3)
			\end{aligned}	
		\]
	\end{proof}
	\subsection*{4.5-10}
	\begin{proof}
		对于$n=12$的基本情况,$T(12)=2T(1)+n\le cn$\\
		假定
		$$T\left(\frac{n}{3}-3\right)le c\left(\frac{n}{3}-3\right) $$
		则
		\[
			\begin{aligned}
				T(n)&\le 2c\left(\frac{n}{3}-3\right)+n\\
				&=\left(\frac{2}{3}c+1\right)n-6c\\
				&\le cn\\
				\therefore T(n)&=O(n)
			\end{aligned}	
		\]
	\end{proof}
	
\end{document}