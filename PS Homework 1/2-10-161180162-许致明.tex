\documentclass[twocolumn]{article}
\usepackage{verbatim}
\usepackage{amsfonts}
\usepackage{geometry}
\usepackage{amsmath}
\usepackage{amsthm}
\usepackage{amssymb}
\usepackage{listings}
\usepackage{graphicx}
\usepackage{clrscode3e}
\usepackage{txfonts}
\usepackage{enumerate}
\usepackage{ctex}
\usepackage{txfonts}
\usepackage{fontspec-xetex}
\usepackage{float}
\geometry{top=2.5cm,bottom=2.5cm,left=2.5cm,right=2.5cm}
\setlength\parindent{0em}
\setmainfont{Times New Roman}
\newcommand*{\dif}{\mathop{}\!\mathrm{d}}
\begin{document}
	\title{问题求解(二)作业(第十周)}\author{161180162 许致明}\maketitle
	\section*{MA第二章}
	\subsection*{2.6}
	对于\proc{Push}操作,操作前$|elemes|<max$,则操作后$|elemes|\le max$满足不变式3;\par 
	操作后,当前的时间戳$c'$满足$c'>c$,其中$c$为栈中元素的时间戳,满足不变式2;\par 
	进行\proc{Push}操作后,加入了一个元素$a$,且当前时间戳增大,故满足不变式1。
	\section*{TC第十章}
	\subsection*{10.1-4}
	\begin{codebox}
		\zi\proc{Enqueue(\id{Q}, \id{x})}\li 
		\If $\id{Q.head}==\id{Q.tail}+1$, or\zi
		$\id{Q.head}==1$ and $\id{Q.tail}=\id{Q.length}$
		\Then\li
		\textbf{error} "overflow" \End \li 
		$\id{Q[Q.tail]}=x$\li 
		\If $\id{Q.tail}==\id{Q.length}$\Then\li 
		$\id{Q.tail}=1$\li 
		\Else\li 
		$\id{Q.tail}=\id{Q.head}+1$
	\end{codebox}
	\begin{codebox}
		\zi\proc{Dequeue(\id{Q}, \id{x})}\li 
		\If $\id{Q.tail}==\id{Q.head}$
		\Then\li
		\textbf{error} "underflow" \End \li 
		$\id{x}=\id{Q[Q.tail]}$\li 
		\If $\id{Q.head}==\id{Q.length}$\Then\li 
		$\id{Q.head}=1$\li 
		\Else\li 
		$\id{Q.head}=\id{Q.head}+1$\End\li 
		\Return $\id{x}$
	\end{codebox}
	\subsection*{10.1-5}
	\begin{codebox}
		\zi\proc{Head-Enqueue(\id{Q}, \id{x})}\li 
		$\id{Q}.[\id{Q.head}]=x$\li 
		\If $\id{Q.head}==1$\Then\li 
		$\id{Q.head}=\id{Q.length}$\li 
		\Else \li 
		$\id{Q.head}=\id{Q.head}-1$\End
	\end{codebox}
	\begin{codebox}
		\zi\proc{Tail-Enqueue(\id{Q}, \id{x})}\li 
		$\id{Q}.[\id{Q.tail}]=x$\li 
		\If $\id{Q.tail}==\id{Q.length}$\Then\li 
		$\id{Q.tail}=1$\li 
		\Else \li 
		$\id{Q.tail}=\id{Q.tail}+1$\End
	\end{codebox}
	\begin{codebox}
		\zi\proc{Head-Dequeue(\id{Q}, \id{x})}\li 
		$\id{x}\gets \id{Q}.[\id{Q.head}]$\li 
		\If $\id{Q.head}==\id{Q.length}$\Then\li 
		$\id{Q.head}=1$\li 
		\Else\li 
		$\id{Q.head}=\id{Q.head}+1$\End\li 
		\Return \id{x}
	\end{codebox}
	\begin{codebox}
		\zi\proc{Tail-Dequeue(\id{Q}, \id{x})}\li 
		$\id{x}\gets \id{Q}.[\id{Q.tail}]$\li 
		\If $\id{Q.tail}==1$\Then\li 
		$\id{Q.tail}=\id{Q.length}$\li 
		\Else\li 
		$\id{Q.tail}=\id{Q.tail}-1$\End\li 
		\Return \id{x}
	\end{codebox}
	\subsection*{10.1-6}
	将进队的操作转化为压栈(假定存入栈$A$),出队时,先将此元素前(在栈的上面)的所有元素弹出,放入另一个栈中(假定为栈$B$),再将此元素弹出。最后把$B$中的元素再按顺序全部出栈压入$A$中。\par 
	此种实现入队操作复杂度为$O(1)$,出队操作复杂度为$O(n)$,$n$为队列中所有元素的个数。
	\subsection*{10.2-1}
	\begin{codebox}
		\zi\proc{Insert(\id{L}, \id{x})} \Comment{$O(1)$}\li 
		$\id{x.next}=\id{L.head}$\li 
		$\id{L.head}=x$
	\end{codebox}
	\begin{codebox}
		\zi\proc{Delete(\id{L}, \id{x})} \Comment{至少为线性时间}\li 
		$\id{p}=\id{L.head}$\li 
		\While $\id{p}\neq x$ and $\id{p}\neq\const{null}$\Do\li 
		$\id{p}=\id{p.next}$\End \li 
		\If $\id{p}==\const{null}$\Then\li 
		\textbf{error} "No such element $x$"\End\li 
		\If $\id{p.prev}\neq \const{null}$\Then\li 
		$\id{p.prev.next}=\id{p.next}$\li 
		\Else \li 
		$\id{L.head}=\id{p.next}$\End\li 
		\If $\id{p.next}\neq \const{null}$\Then\li 
		$\id{p.next.prev}=\id{p.prev}$\End\li 
		\Return $\id{p}$
	\end{codebox}
	\subsection*{10.2-2}
	\begin{codebox}
		\zi\proc{Push(\id{L}, \id{x})}\li 
		$\id{x.next}=\id{L.head}$\li 
		$\id{L.head}=\id{x}$
	\end{codebox}
	\begin{codebox}
		\zi\proc{Pop(\id{L}, \id{x})}\li 
		$\id{x}=\id{L.head}$\li 
		$\id{L.head}=\id{x.next}$\li 
		\Return $\id{x}$
	\end{codebox}
	\subsection*{10.2-3}
	加入哨兵:$\id{L.head}\rightarrow\id{L.\const{null}}$(哨兵)$\rightarrow\id{L.end}$
	\begin{codebox}
		\zi\proc{Dequeue(\id{L}, \id{x})}\li 
		$\id{x}=\id{L.head}$\li 
		$\id{L.head}=\id{x.next}$\li 
		\Return $\id{x}$
	\end{codebox}
	\begin{codebox}
		\zi\proc{Enqueue(\id{L}, \id{x})}\li 
		$\id{x.next}=\id{L.\const{null}.next}$\li 
		$\id{L.\const{null}.next}=\id{x}$
	\end{codebox}
	\subsection*{10.2-6}
	设$L_1$为包含了$S_1$中所有元素的双向链表,$L_2$为包含了$S_2$中所有元素的双向链表。
	\begin{codebox}
		\zi\proc{Union($\id{L_1}$, $\id{L_2}$)}\li 
		$\id{L_1.\const{null}.prev.next}=\id{L_2.\const{null}.next}$\li 
		$\id{L_2.\const{null}.next.prev}=\id{L_1.\const{null}.prev}$\li
		$\id{L_1.\const{null}.prev}=\id{L_2.\const{null}.prev}$\li 
		$\id{L_2.\const{null}.prev.next}=\id{L_1.\const{null}.prev}$\li 
	\end{codebox}
	\subsection*{10.3-4}
	$m$指示已经分配的元素个数。
	\begin{codebox}
		\zi\proc{Allocate-Object(\id{A})}\li 
		\If $\id{m}==\id{A.length}$\Then\li 
		\textbf{error} "out of space"\li 
		\Else \li 
		$\id{m}=\id{m}+1$\li 
		$\id{x}=\id{m}$
		\End\li 
		\Return \id{x}
	\end{codebox}
	\begin{codebox}
		\zi\proc{Free-Object(\id{A}, \id{x})}\li 
		\If $\id{m}==0$\Then\li 
		\textbf{error} "underflow"\li 
		\ElseIf $\id{x}==m$\Then \li 
		$\id{m}=\id{m}-1$\li 
		\Else\li 
		\proc{Swap(\id{x}, \id{m})}\li 
		$\id{m}=\id{m-1}$\End 
	\end{codebox}
	\subsection*{10.3-5}
	\begin{codebox}
		\zi\proc{Compactify-List(\id{L}, \id{F})}\li 
		$\id{p_1}=1,\ \id{p_2}=\id{A.length}$\li 
		\While $\id{p_1}<\id{p_2}$\Do\li 
		\While $\id{key}[\id{p_1}]\neq \const{null}$\Do\li 
		$\id{p_1}=\id{p_1}+1$\End\li 
		\While $\id{key}[\id{p_2}]==\const{null}$\Do\li 
		$\id{p_2}=\id{p_2}-1$\End\li 
		\If $\id{p_1}<\id{p_2}$\Then\li 
		\proc{Swap($\id{key}[\id{p_1}]$, $\id{key}[\id{p_2}]$)}\li 
		\For $\id{i}=1$ \To $\id{A.length}$\Do\li 
		\If $\id{next}[\id{i}]==p_2$\Then\li 
		$\id{next}[\id{i}]==p_1$\End\li 
		\If $\id{prev}[\id{i}]==p_2$\Then\li 
		$\id{prev}[\id{i}]==p_1$\End\li
		\If $\id{next}[\id{i}]==p_1$ and $\id{key}[\id{i}]==\const{null}$\Then\li 
		$\id{next}[\id{i}]==p_2$\End
	\end{codebox}
	使用循环不变量:$p_1$前存放的都是数字,$p_2$后全部为空,可以证明算法的正确性。
	\subsection*{10.4-2}
	\begin{codebox}
		\zi\proc{Print-Tree(\id{T})}\li 
		\If $\id{T}\neq \const{null}$\Then\li 
		print $\id{T}.key$\li 
		\proc{Print-Tree(\id{T.left})}\li 
		\proc{Print-Tree(\id{T.right})}
	\end{codebox}
	\subsection*{10.4-3}
		\begin{codebox}
			\zi\proc{Print-Tree-Stack(\id{T})}\li 
			\proc{Push(\id{S}, \id{T})}\li 
			\While $\id{S}\neq \const{null}$\Do\li 
			$\id{tmp}=\proc{Pop(\id{S})}$\li
			print $\id{tmp}.\id{key}$\li 
			\If $\id{tmp.left}\neq \const{null}$\Then\li 
			\proc{Push(\id{S}, \id{tmp.left})}\End\li  
			\If $\id{tmp.right}\neq \const{null}$\Then\li 
			\proc{Push(\id{S}, \id{tmp.right})}\End
		\end{codebox}
	\subsection*{10.4-4}
		\begin{codebox}
			\zi\proc{Print-Tree-Arbitrary(\id{T})}\li 
			\proc{Push(\id{S}, \id{T})}\li 
			\While $\id{S}\neq \const{null}$\Do\li 
			$\id{tmp}=\proc{Pop(\id{S})}$\li
			print $\id{tmp}.\id{key}$\li 
			\If $\id{tmp.left-child}\neq \const{null}$\Then\li 
			\proc{Push(\id{S}, \id{tmp.left-child})}\End\li  
			\If $\id{tmp.right-sibling}\neq \const{null}$\Then\li 
			\proc{Push(\id{S}, \id{tmp.right-sibling})}\End
		\end{codebox}
	\subsection*{10-3}
	\begin{enumerate}[(a)]
		\item 若算法二在$1-t$的循环中已经返回,则与算法一等价,此时迭代的总次数为$i$;\par 
		否则,在循环结束后,必有$\id{key}[\id{i}]<\id{key}[\id{j}]$,且$\id{key}[\id{j}]<k$。则接下来的8、9行运行中进行搜索,可以得到正确的结果。次数算法二的总迭代次数大于$t$。
		\item 若在$1-t$的循环中返回,则运行时间为$O(t)$;\par 
		否则,在\For 循环结束后的过程中,期望运行时间为$E[X_t]$。因此,总的期望运行时间为$O(t+E[X_t])$。
		\item 
		\[
			\begin{aligned}
				E[X_t]&=\sum_{i=0}^{d}iP(X_t=i)=\sum_{i=1}^{d}P(X_t\ge i)\\
				&\le \sum_{i=1}^{n}P(X_t\ge i)\le \sum_{i=1}^{n}\left(1-\frac{r}{n}\right)^t
			\end{aligned}
		\]
		\item 
		\[
			\begin{aligned}
				\sum_{i=0}^{n-1}r^t\le \int_{0}^{n}x^t\dif x=\frac{n^{t+1}}{t+1}
			\end{aligned}
		\]
		\item 
		\[
			\begin{aligned}
				E[X_t]&=\sum_{i=0}^{n}\left(1-\frac{i}{n}\right)^t=\frac{1}{n^t}\sum_{i=0}^{n-1}i^t\\
				&\le \frac{1}{n^t}\cdot \frac{n^{t+1}}{t+1}=\frac{n}{t+1}
			\end{aligned}
		\]
		\item 
		\[
			\begin{aligned}
				O(t+E[X_t])=O\left(t+\frac{n}{t+1}\right)=O\left(t+\frac{n}{t}\right)
			\end{aligned}
		\]
		\item 
		\[
			\begin{aligned}
				t+\frac{n}{t}&\ge 2\sqrt{t\cdot \frac{n}{t}}=2\sqrt{n}\\
			\end{aligned}
		\]
		故当$t=\sqrt{n}$时,期望运行时间取最小,为$O(\sqrt{n})$。
		\item 
		当所有元素均相同,且不是所要找的元素时,渐进运行时间无法降低。
	\end{enumerate}
\end{document}