\documentclass{article}
\usepackage{verbatim}
\usepackage{amsfonts}
\usepackage{geometry}
\usepackage{amsmath}
\usepackage{amsthm}
\usepackage{amssymb}
\usepackage{listings}
\usepackage{graphicx}
\usepackage{clrscode3e}
\usepackage{txfonts}
\usepackage{enumerate}
\usepackage{ctex}
\usepackage{txfonts}
\usepackage{fontspec-xetex}
\usepackage{float}
\geometry{top=2.5cm,bottom=2.5cm,left=2.5cm,right=2.5cm}
\setlength\parindent{0em}
\setmainfont{Times New Roman}
\newcommand*{\dif}{\mathop{}\!\mathrm{d}}
\begin{document}
	\title{问题求解(二)作业(第九周)}\author{161180162 许致明}\maketitle
	\section*{TC第七章}
	\subsection*{7.1-3}
	\begin{proof}
		\For 循环共执行了$r-p$次,而每次执行消耗$\Theta(1)$时间,而循环外的操作消耗时间也为$\Theta(1)$。因此\proc{Partition}消耗$\Theta(r-p)$时间,即$\Theta(n)$,$n$为子数组的大小。
	\end{proof}
	\subsection*{7.2-4}
	\begin{proof}
		假定该数组中有$c$个元素是未排列好的,$c$为与数组大小$n$无关的常数。对于插入排序,循环次数为$c$,则运行时间为$O(cn)=O(n)$。而对快速排序,这种输入将会产生最坏情况划分,导致最坏运行时间$O(n^2)$。因此对于此类输入,插入排序效果好于快速排序。
	\end{proof}
	\subsection*{7.3-2}
	最坏情况:$\Theta(n)$\par 
	最好情况:$\Theta(n)$
	\subsection*{7.4-2}
	设最好情况运行时间为$T(n)$,则:
	\[
	\begin{aligned}
		T(n)&=\min_{1\le q\le n-1}\left(T(q)+T(n-q-1)\right)+O(n)\\
		\text{假设$T(n)\ge c\left(n\log n+2n\right)$,$c$为常数}\\
		T(n)&\ge \min_{1\le q\le n-1}\left(cq\log q+2cq+c\left(n-q-1\right)\log \left(n-q-1\right)+2c\left(n-q-1\right)\right)+\Theta(n)\\
		&=\frac{cn}{2}\log (n/2)+cn+c(n/2-1)\log(n/2-1)+cn-2c+\Theta(n)\\
		&\ge cn\log n+cn/2-\log n+2-2c+\Theta(n)
	\end{aligned}
	\]
	当$q=n/2$时,上式最小,得$T(n)\ge cn\log n$,即$T(n)=\Omega (n\log n)$
	\subsection*{7-5}
	\begin{enumerate}[a.]
		\item 
		$$p_i=\frac{6(n-i)(i-1)}{n(n-1)(n-2)}$$
		\item 
		\[
		\begin{aligned}
			i&=\left\lfloor\frac{n+1}{2}\right\rfloor\\
			\text{则结果为:}f(n,i)&=\frac{6\left(\left\lfloor\frac{n-1}{2}\right\rfloor\right)\left(n-\left\lfloor\frac{n+1}{2}\right\rfloor\right)}{n(n-1)(n-2)}-\frac{1}{n}\\
			\lim\limits_{n\rightarrow \infty}f(n,i)&=\frac{3}{2}
		\end{aligned}
		\]
		\item 设$n$为3的幂,则:
		\[
		\begin{aligned}
		S(n)=\sum_{i=n/3}^{2n/3}p_i&\approx \int_{n/3}^{2n/3}\frac{6(-x^2+nx+x-n)}{n(n-1)(n-2)}\dif x\\
		\lim\limits_{n\rightarrow \infty}S(n)&=\frac{13}{27}>\frac{1}{3}
		\end{aligned}
		\]
		\item 三数取中法能得到更好的划分。但是根据7.4节内容,任何常数比例的划分都会产生$\Theta(\log n)$的递归树,且每层的代价为$\Theta(n)$。因此仍有$\Omega(n\log n)$的复杂度下界,但更好的划分可以减小隐藏在$\Omega$符号中的常数。
		\end{enumerate}
		\section*{TC第八章}
		\subsection*{8.1-4}
		对于整个输入序列,共有$k!^{n/k}$种输出的可能,设搜索树的高度为$h$,则:
		\[
		\begin{aligned}
			k!^{n/k}&\le 2^h\\
			h\ge (n/k)\log (k!)&\ge(n/k)\left(\frac{k\ln k}{\ln 2}\right)=\Omega(n\log k) 
		\end{aligned}
		\]
	\subsection*{8.2-4}
	执行计数排序的1-9行,然后返回$C[b]-C[a-1]$。
	\subsection*{8.3-4}
	将所有数字转换为$n$进制数,然后使用基数排序。
	\subsection*{8.2-4}
	当所有元素都在同一个桶中出现最坏情况,这时需要$O(n^2)$的运行时间。解决方法:对每个桶都使用归并排序,这样最坏复杂度为$O(n\log n)$
	\subsection*{8-2}
	\begin{enumerate}[a.]
		\item 
		\begin{codebox}
			\zi\proc{Sort-a(\id{A})}\li 
			Let \id{C} be a new array\li 
			\id{index}=1\li 
			\For $\id{i}=\gets 1$ \To $n$\Do\li 
			\If \id{A[i]}==0\Then\li 
			\id{C[index]}=A[i]\li 
			\id{index}=\id{index}+1\End\End\li 
			\For $\id{i}=\gets 1$ \To $n$\Do\li 
			\If \id{A[i]}==1\Then\li 
			\id{C[index]}=A[i]\li 
			\id{index}=\id{index}+1\End\End\li
			\Return \id{C} 
		\end{codebox}
	\item 
	\begin{codebox}
		\zi\proc{Sort-b(\id{A})}\li 
		\id{index}=1\li 
		\For \id{i}=1 \To \id{n}\Do\li 
		\If \id{A[i]}==0\Then\li 
		\proc{Swap(\id{A[i]},\id{A[index]})}\li 
		\id{index}=\id{index}+1\End\End\li 
		\Return \id{A}
	\end{codebox}
	\item 冒泡排序
	\item 使用\proc{Sort-a}
	\item 先运行计数排序的$1-5$行,然后将得到的数字复制一份,运行$7-9$行
	\end{enumerate}
	\subsection*{9.1-1}
	\begin{enumerate}[(1)]
		\item 	$n-1$次比较找到最小值
		\item 在大于最小值的$\lceil\log n\rceil$个数中找到次小值,比较$\lceil\log n\rceil-1$次
	\end{enumerate}
	比较次数$n+\lceil\log n\rceil-2$
	\subsection*{9.3-7}
	\begin{enumerate}[(1)]
		\item 找到中位数
		\item 每个数减去中位数,取绝对值
		\item 找到第$k$小的数
		\item 选取绝对值小于等于第$k$小的数的所有数
	\end{enumerate}
\end{document}