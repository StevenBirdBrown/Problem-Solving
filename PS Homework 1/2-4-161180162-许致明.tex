\documentclass[twocolumn]{article}
\usepackage{verbatim}
\usepackage{amsfonts}
\usepackage{geometry}
\usepackage{amsmath}
\usepackage{amsthm}
\usepackage{amssymb}
\usepackage{listings}
\usepackage{graphicx}
\usepackage{clrscode3e}
\usepackage{txfonts}
\usepackage{enumerate}
\usepackage{ctex}
\usepackage{txfonts}
\usepackage{fontspec-xetex}
\usepackage{float}
\geometry{top=2.5cm,bottom=2.5cm,left=2.5cm,right=2.5cm}
\setlength\parindent{0em}
\setmainfont{Times New Roman}
\begin{document}
	\title{问题求解(二)作业(第四周)}\author{161180162 许致明}\maketitle
	\section*{TC第四章}
	\subsection*{4.1-5}
		\begin{codebox}
			\zi \proc{Max-Subarray$(\id{A[1..n]})$}\li 
			$\id{max}\gets -\infty,\id{tmax}\gets 0,\id{s}\gets \const{null},\id{e}\gets \const{null},\id{ind}\gets 1$\li
			\For $\id{i}\gets 1$\To $\id{n}$\Do\li 
			$\id{tmax}\gets \id{tmax}+A[i]$\li  
			\If $\id{tmax}>\id{max}$\Then \li 
			$\id{s}\gets \id{ind}$\li
			$\id{e}\gets \id{i}$\li 
			$\id{max}\gets \id{tmax}$ 
			\End\li 
			\If $\id{tmax}<0$\Then\li 
			$\id{tmax}\gets 0$\li 
			$\id{ind}\gets\id{i}+1$ 
			\End
			\End\li 
			\Return $(\id{s},\id{e},\id{max})$
		\end{codebox}
	\subsection*{4.3-3}
	\begin{proof}
		设$T(n)\ge cn\log n$,其中$c=\min(1/3,T(2)/2)$。则有:
		\[
		\begin{aligned}
			T(n)&=2T(\rfloor n/2\lfloor)+n\ge 2c\lfloor n/2\rfloor\log (\lfloor n/2\rfloor)+n\\
			&\ge c(n-1)\log((n-1)/2)+n\\
			&=c(n-1)(\log n-1-\log(n/(n-1)))+n\\
			&=cn\left(\log n-1-\log(n/(n-1))+\frac{1}{c}\right)\\
			&-c\log n-1-\log (n/(n-1))\\
			&\ge cn(\log n-C),\text{ ($C$为常数)}\\
			&\ge cn\log n\\
			\therefore T(n)&=\Omega(n\log n)
		\end{aligned}
		\]
	\end{proof}
	\subsection*{4.4-2}
	根据递归树猜测$T(n)=n^2$,下面证明此结论。
	\begin{proof}
		\[
		\begin{aligned}
			T(n)&=T(n/2)+n^2\\
			&\le c(n/2)^2+n^2\\
			&=\left(\frac{c}{4}+1\right)n^2\\
			&\le Cn^2\text{ ($C$为常数)}
		\end{aligned}
		\]
	\end{proof}
	\subsection*{4.5-3}
	参考主定理的形式,有$a=1,b=2$。则:$$n^{\log_b a}=n^{\log_2 1}=1.$$递归式解为第二种形式,即$$\Theta (\log n).$$
	\subsection*{4-4}
	\begin{enumerate}[a.]
		\item 
		\[
		\begin{aligned}
				\because &F_0=0,F_1=1,F_i=F_{i-1}+F_{i-2},\forall i\ge 2\\
				\therefore\mathcal{F}(z)&=\sum_{i=0}^{\infty}F_i{z}^i\\
				&=F_0+F_1+\sum_{i=2}^{\infty}(F_i-1+F_{i-2}){z}^i\\
				&={z}+{z}\sum_{i=2}^{\infty}F_{i-1}{z}^{i-1}+{z}^2\sum_{i=2}^{\infty}F_{i-2}{z}^{i-2}\\
				&={z}+{z}\sum_{i=1}^{\infty}F_i{z}^i+{z}^2\sum_{i=0}^{\infty}F_i{z}^i\\
				&={z}+{z\mathcal{F}(z)}+{z}^2{\mathcal{F}(z)}.
		\end{aligned}
		\]
		\item
		\[
			\begin{aligned}
			\because \mathcal{F}(z)-z\mathcal{F}(z)-z^2\mathcal{F}(z)=z\\
			\therefore \mathcal{F}(z)=\frac{z}{1-z-z^2}\\
			\therefore 1-z-z^2=(1-\phi z)(1-\hat{\phi}z)
			\end{aligned}
		\]
		\item 
		\[
		\begin{aligned}
			\mathcal{F}(z)&=\frac{1}{\sqrt{5}}\left(\frac{1}{1-\phi z}-\frac{1}{1-\hat{\phi}z}\right)\\
			&=\frac{1}{\sqrt{5}}\left(\sum_{i=0}^{\infty}(\phi z)^i-\sum_{i=0}^{\infty}(\hat{\phi}z)^i\right)\\
			&=\sum_{i=0}^{\infty}\frac{1}{\sqrt{5}}\left(\phi^i-\hat{\phi}^i\right)z^i
		\end{aligned}
		\]
		\item 
		\[
			\begin{aligned}
			\because\left|\hat{\phi}\right|<1|\\
			\therefore\left|\frac{\hat{\phi}^i}{5}\right|(i\ge 2)< \left|\frac{\hat{\phi}}{5}\right|<\frac{1}{2}
			\end{aligned}	
		\]
		又因为斐波那契数均为整数,所以$F_i=\lfloor\phi^i/\sqrt{5}\rfloor.$
	\end{enumerate}
\end{document}