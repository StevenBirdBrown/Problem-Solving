\documentclass[twocolumn]{article}
\usepackage{verbatim}
\usepackage{amsfonts}
\usepackage{geometry}
\usepackage{amsmath}
\usepackage{amsthm}
\usepackage{amssymb}
\usepackage{listings}
\usepackage{graphicx}
\usepackage{clrscode3e}
\usepackage{txfonts}
\usepackage{enumerate}
\usepackage{ctex}
\usepackage{txfonts}
\usepackage{fontspec-xetex}
\usepackage{float}
\geometry{top=2.5cm,bottom=2.5cm,left=2.5cm,right=2.5cm}
\setlength\parindent{0em}
\setmainfont{Times New Roman}
\newcommand*{\dif}{\mathop{}\!\mathrm{d}}
\begin{document}
	\title{问题求解(二)作业(第十一周)}\author{161180162 许致明}\maketitle
	\section*{TC第六章}
	\subsection*{6.1-2}
	\[
	\begin{aligned}
		\exists p,s.t.\ n+p=2^h-1\\
		h=\log(n+p+1)\\
		\therefore h<\log n\\
		\text{又}\because h>h-1\\
		\therefore h>\log 2^{n-1}\\
		h>\log \frac{n+p+1}{2}>\log\frac{n}{2}\\
		\therefore h>\log -1\\
		\text{综合得,}\log n-1<h<\log n\\
		\therefore n=\lfloor\log n\rfloor
	\end{aligned}
	\]
	\subsection*{6.1-4}
	在叶节点中,即数组下标为$\lceil n/2\rceil\sim n$的元素中。
	\subsection*{6.1-7}
	对于下标在$0\sim \lfloor n/2\rfloor$的元素,求出的下标仍在原数组中。而对$\lfloor n/2\rfloor+\sim n$的元素,求出的孩子下标越界,即它们都是叶节点。
	\subsection*{6.2-2}
	\begin{codebox}
		\zi\proc{Min-Heapify(\id{A}, \id{i})}\li 
		\id{l}$\gets $\proc{Left(\id{i})}\li
		\id{r}$\gets $\proc{Right(\id{i})}\li 
		\If \id{l}$\le $\id{A}.\id{heapsize} and $A[l]<A[i]$ \Then\li 
		\id{smallest}$\gets l$
	\end{codebox}
	\begin{codebox}
		\li \Else\Then\li 
		\id{smallest}$\gets i$\End\li 
		\If $r\le A.heapsize$ and $A[r]<A[i]$\Then\li 
		\id{smallest}$\gets r$\End\li 
		\If \id{smallest}$\neq i$\Then\li 
		\proc{Swap(\id{A[i]}, \id{A[smallest]})}\li 
		\proc{Min-Heapify(\id{A}, \id{smallest})}\End
	\end{codebox}
	运行时间不变。
	\subsection*{6.2-5}
	\begin{codebox}
		\zi\proc{Max-Heapify(\id{A}, \id{i})}\li 
		\While \const{true}\Do\li 
		$\id{l}\gets \proc{Left(\id{i})}$\li 
		$\id{r}\gets \proc{Right(\id{i})}$\li 
		\If \id{l}$\le $\id{A}.\id{heapsize} and $A[l]>A[i]$ \Then\li 
		\id{largest}$\gets l$\li 
		\Else\li 
		\id{largest}$\gets i$\End\li 
		\If $r\le A.heapsize$ and $A[r]>A[i]$\Then\li 
		\id{largest}$\gets r$\End\li 
		\If \id{largest}$\neq i$\Then\li 
		\proc{Swap(\id{A[i]}, \id{A[largest]})}\li 
		\id{i}$\gets$\id{largest}\End
	\end{codebox}
	\subsection*{6.2-6}
	将最小值放置在根节点,且对于每个非叶节点,保证左孩子大于右孩子。对此堆的根做\proc{Max-Heapify},共需$h$次($h$为树的高度),则$h=\lfloor \log n\rfloor$。且每次运行均消耗常数时间,故最坏情况的运行时间为$\Omega(\log n)$
	\subsection*{6.3-3}
	\begin{enumerate}[(1)]
		\item 奠基:
		$$h=1,\lceil n/2^2\rceil=\lceil 1/4\rceil=1$$
		\item 归纳假设:
		$$h=k,\ \text{高度为$h$的节点数量$m$不大于}\lceil n/2^{k+1}\rceil$$
		\item $$h=k+1,\ m\le \lceil n/2^{(h+1)+1}\rceil=\lceil n/2^{h+2}\rceil$$
		故至多有$\lceil n/2^{h+1}\rceil$个高度为$h$的节点。
	\end{enumerate}
	\subsection*{6.4-2}
	\begin{proof}
	\begin{enumerate}[(1)]
		\item 初始:$i=A.length$,最大堆已经建好,$A[1..A.length]$包含所有元素,$A[i+1..n]$包含$0$个最大元素,$n-A.length=0$。成立;
		\item 保持: 假设在第$n-i$次循环前,$A[1..i]$含全部元素中第$i$小的,$A[i+1..n]$含已排序的$n-i$个最大元素。在第$i$次循环中,此时$A[1]$是第$(n-i+1)$大的元素,也是第$i$小的元素。将它与$A[i]$交换,使得$A[1..n]$含已排序的$n-i+1$个最大元素。由此可知,第$n-1$小的元素在$A[1..i-1]$中,故在第$n-(i-1)$次循环前,不变式成立;
		\item 终止:循环进行$n-1$次后停止,此时$i=2$,由不变式,$A[2..n]$中是$n-1$个最大元素排列好的序列,只有最小的元素在首位。因此,整个数组已经排好序。
	\end{enumerate}
	\end{proof}
	\subsection*{6.4-4}
	\[
	\begin{aligned}
		T(\proc{Heapsort})&=T(\proc{Build})+n\cdot T(\proc{Max-Heapify})\\
		T(\proc{Max-Heapify})&=\Omega(\log n)\\
		T(\proc{Heapsort})&=\sum_{i=1}^{n-1}\Omega(\log i)=\Omega(\log (n-1)!)\\
		&=\Omega(n\log n)
	\end{aligned}
	\]
	\subsection*{6.5-5}
	\begin{proof}
		\begin{enumerate}[(1)]
			\item 初始:由假定,满足不变式;
			\item 保持:假设第$t$次循环前,满足不变式。
			\begin{enumerate}[1.]
				\item 若此时$A[1..A.length]$满足最大堆的性质,则$A[\proc{Parent(i)}]>A[i]$,退出循环,不变式依然满足;
				\item 
				若此时$A[i]>A[\proc{Parent(i)}]$,则进入循环,并将两者调换位置,再使得$i$成为父节点。调换后堆其他位置的性质不变,以原本的子节点为根的树满足最大堆性质,此时不确定的只有原本的父节点的位置,即现在的$A[i]$。
			\end{enumerate}
		故每次循环能保持不变式;
		\item 终止:每次循环后有两种情况可以终止:
		$i\le 1$或$Ap\proc{Parent(i)}>A[i]$。当后者满足时,由不变式可得$A[1..A.length]$是最大堆。若后者不满足,每次将$i$的值减半,最终使得$i\le 1$。前者满足时,$i$的父节点不存在,不会违背条件。故$A$是最大堆。
		\end{enumerate}
	\end{proof}
	\subsection*{6.5-7}
	FIFO队列:将入队(放入优先队列)的时间作为value,建立value小优先的优先队列;\par 
	栈:将入栈(放入优先队列)的时间作为value,建立value大优先的优先队列。
	\subsection*{6.5-9}
	利用每个链表头部的元素(共$k$)个建立最小堆,每次取出堆里最小的元素。然后从取出元素所在的链表中拿出下一个元素添加进最小堆(前述过程的时间复杂度为$O(\log k)$)。如此重复,直至$k$个链表中的元素全部被取出。因此,总时间复杂度为$n\cdot O(\log k)=O(n\log k)$。
\end{document}