\documentclass[twocolumn]{article}
\usepackage{verbatim}
\usepackage{amsfonts}
\usepackage{geometry}
\usepackage{amsmath}
\usepackage{amsthm}
\usepackage{amssymb}
\usepackage{listings}
\usepackage{graphicx}
\usepackage{clrscode3e}
\usepackage{txfonts}
\usepackage{enumerate}
\usepackage{ctex}
\usepackage{txfonts}
\usepackage{fontspec-xetex}
\usepackage{float}
\geometry{top=2.5cm,bottom=2.5cm,left=2.5cm,right=2.5cm}
\setlength\parindent{0em}
\setmainfont{Times New Roman}
\begin{document}
	\title{问题求解(二)作业(第七周)}\author{161180162 许致明}\maketitle
	\section*{CS第五章}
	\subsection*{5.1.10}
	\begin{enumerate}[(1)]
		\item 可将52张牌视作13张不同牌重复四次得到,则形成顺子需要在这13张牌中选出连续的5张。这种情况共有$13-5+1=9$种。故加上这5张牌的花色,共有$9\times 4^5=9216$种方法。
		\item 记此为事件$A$,则:
		\[
		\begin{aligned}
		P(A)=\frac{9216}{\binom{52}{5}}=0.0035
		\end{aligned}
		\]
	\end{enumerate}
	\subsection*{5.1.12}
	记两次顶图案相同为事件$A$,则$A$发生时,顶部均为正方形、圆或三角形。
	\[
	P(A)=\left(\frac{1}{6}\right)^2+\left(\frac{2}{6}\right)^2+\left(\frac{3}{6}\right)^2=\frac{7}{18}
	\]
	\subsection*{5.2.4}
	设至少有一个是红色或白色为事件$A$,则$A$的反面是没有任何一个球是红色且没有任何一个球是白色。显见$P(\overline{A})=0$,故$P(A)=1-P(\overline{A})=1$。\par 
	设至少有一个球为红色为事件$B$,则$B$的反面是没有任何一个球是红色,这种情况仅发生在取走的两个球均为红色时,故:
	\[
	P(B)=\frac{1}{\binom{6}{2}}=\frac{1}{15}
	\]
	\subsection*{5.2.10}
	设每个地点有至少一个值为事件$A$,显见当$k>n$时,$A$不可能发生,$P(A)=0$;\par 
	当$k\le n$时,$P(A)=\sum_{i=0}^{k}(-1)^i\binom{k}{i}(k-i)^n$
	\subsection*{5.3.2}
	记连续两次出现正面为事件$A$,正面向上的次数为偶数为事件$B$,则:
	\[
	\begin{aligned}
		P(A)&=\left(\frac{1}{2}\right)^3+\binom{2}{1}\left(\frac{1}{2}\right)^3\\
		&=\frac{3}{8}\\
		P(B)&=\binom{3}{0}\left(\frac{1}{2}\right)^3+\binom{3}{2}\left(\frac{1}{2}\right)^3\\
		&=\frac{1}{2}\\
		P(A|B)&=P(A\cap B)/P(B)\\
		&=\frac{2}{8}/\frac{1}{2}\\
		&=\frac{1}{2}\neq P(A)\\
	\end{aligned}
	\]
	故{这两个事件不互相独立。}
	\subsection*{5.3.6}
	记面答对任意的一道题为事件$A$,则:
	\[
	P(A)=60\%\times 1+40\%\times \frac{1}{2}=\frac{4}{5}
	\]
	记此事件为$B$,则:
	\[
	\begin{aligned}
		P(B)&=60\%+40\%\times\left(\frac{1}{2}\times\frac{1}{2}\times 2\right)\\
		&=\frac{4}{5}
	\end{aligned}
	\]
	\subsection*{5.3.12}
	记男孩为$b$,女孩为$g$,则所有可能为(先出现的年龄大):
	\[
	\left\{(b,b),(b,g),(g,g),(g,b)\right\}
	\]
	记有两个女孩为事件$A$,其中一个是女孩为事件$B$,则:
	\[
	\begin{aligned}
		P(A)&=\frac{1}{4}\\
		P(B)&=\frac{3}{4}\\
		P(A\cap B)&=\frac{1}{4}\\
		\therefore P(A|B)&=P(A\cap B)/P(B)=\frac{1}{3}
	\end{aligned}
	\]
	记有两个男孩为事件$C$,年长的是男孩为事件$D$,则:
	\[
	\begin{aligned}
		P(C)&=\frac{1}{4}\\
		P(D)&=\frac{1}{2}\\
		P(C\cap D)&=\frac{1}{4}\\
		\therefore P(C|D)&=P(C\cap D)/P(D)=\frac{1}{2}
	\end{aligned}
	\]
	\subsection*{5.4.4}
	可知一共有两种序列满足条件,记此事件为$A$,则:
	\[
		P(A)=0.8^4\times 0.2+0.2\times 0.8^4=\frac{512}{3125}
	\]
	记恰好答对4道题为事件$B$,则:
	\[
		P(B)=\binom{5}{4}(0.8)^4\times 0.2=\frac{256}{625}
	\]
	\subsection*{5.4.10}
	\[
		E(c)=\sum X_iP(s_i)=c\sum P(s_i)=c
 	\]
 	\subsection*{5.4.12}
 	不妨设考试共有$N$道题目,此学生掌握了$r$的考试内容($0\le r\le 1$)。则记答对的题目个数为$n$,答错的题目个数为$m$,可得:
 	\[
 	\begin{aligned}
 		n&=N\cdot r+\frac{1}{5}\cdot N\cdot\left(1-r\right)\\
 		&=\frac{N}{5}\left(1+4r\right)\\
 		\therefore m&=N-n\\
 		&=\frac{4N}{5}\left(1-r\right)\\
 		\because n-y\cdot m &=Nr\\
 		\frac{N}{5}\left(1+4r\right)&-y\cdot \frac{4N}{5}\left(1-r\right)=Nr\\
 		y&=\frac{1}{4}
 	\end{aligned}
 	\]
 	\subsection*{5.4.15}
 	\begin{proof}
 		\[
 		\begin{aligned}
 			E(cX)=\sum_{s:s\in S}cX(s)P(s)=c\sum_{s:s\in S}X(s)P(s)=cE(X)
 		\end{aligned}
 		\]
 	\end{proof}
\end{document}