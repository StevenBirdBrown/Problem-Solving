\documentclass[twocolumn]{article}
\usepackage{verbatim}
\usepackage{amsfonts}
\usepackage{geometry}
\usepackage{amsmath}
\usepackage{amsthm}
\usepackage{amssymb}
\usepackage{listings}
\usepackage{graphicx}
\usepackage{clrscode3e}
\usepackage{txfonts}
\usepackage{enumerate}
\usepackage{ctex}
\usepackage{txfonts}
\usepackage{fontspec-xetex}
\usepackage{float}
\geometry{top=2.5cm,bottom=2.5cm,left=2.5cm,right=2.5cm}
\setlength\parindent{0em}
\setmainfont{Times New Roman}
\begin{document}
	\title{问题求解(二)作业(第七周)}\author{161180162 许致明}\maketitle
	\section*{CS第五章}
	\subsection*{5.6.4}
	$P(A)=P(K)=P(Q)=P(J)=\frac{2}{9}$, $P(K)=\frac{1}{9}$. 设赢钱的期望为$X$,则:
	\[
		\begin{aligned}
			&\ \ \ \ \ E[X]\\
			&=P(A)\cdot\left(1+E[X]\right)+P(J)\cdot 2+P(Q)\cdot 3+P(K)\cdot 4\\
			&=\frac{12}{7}
		\end{aligned}
	\]
	故理智人最多花$\frac{12}{7}\$$玩这个游戏。
	\subsection*{5.6.8}
	\begin{proof}
		\[
		\begin{aligned}
			\sum_{i=1}^{n}E[X|F_i]P(F_i)&=\sum_x\sum_{i=1}^{n}xP(X=x|F_i)P(F_i)\\
			&=\sum_x\sum_{i=1}^{n}x\frac{P(X=x\cap F_i)}{P(F_i)}\cdot P(F_i)\\
			&=\sum_x\sum_{i=1}^{n}xP(X=x\cap F_i)\\
			&=\sum_{x\in S}xP(X=x)\ \ \ (\because \bigcup_{i=1}^{n}F_i=S)\\
			&=E[X]
		\end{aligned}
		\]
	\end{proof}
	\subsection*{5.7.2}%4/6/12
	$E[X_i]=P(X_i=1)=0.6$. $X_i\sim B(1,0.6)$, $\therefore D[X]=0.24$. $\sum_{i=1}^{5}X_i=X$,因为$X_1,X_2,\dots,X_5$相互独立。
	\subsection*{5.7.4}
	$E[X]=100\times 0.6=60$, $D[X]=100\times 0.6\times 0.4=24$, $\sigma[X]=\sqrt{D[X]}=2\sqrt{6}$
	\subsection*{5.7.6}
	$25:\ D[X]=25\times 0.8\times 0.2=4$\\
	$100:\ D[X]=16$\\
	$400:\ D[X]=64$\\
	改用标准差。
	\subsection*{5.7.12}
	设题目共有$n$道,$I_i=\left\{\begin{array}{cl}
	1,&\text{第$i$题答对}\\
	0,&\text{否则}
	\end{array}\right.$。则:
	$$P(X_i=1)=4/5,\ D[X_i]=\frac{1}{5}\times\frac{4}{5}=\frac{4}{25}$$
	$$D[X]=\sum_{i=1}^{n}D[X_i]=0.16n$$
	$$\sigma[X]=\sqrt{D[X]}=0.4\sqrt{n}$$
	$$\therefore 2\cdot\sigma[X]=0.05n$$
	$$n=256$$
	\section*{TC第五章}
	\subsection*{5.2.4}
	设$I_i=\left\{\begin{array}{cl}
	1,&\text{第$i$个客人拿到了他的帽子}\\
	0,&\text{否则}
	\end{array}\right.$,则$P(I_i=1)=1/n,\forall i\in\{1,2,\dots,n\}$。设$X$为最终拿到自己的帽子的客人数,则:
	\[
	\begin{aligned}
		E[X]=\sum_{i=1}^{n}E[I_i]=\sum_{i=1}^{n}P(I_i)=1
	\end{aligned}
	\]
	\subsection*{5.2.5}
	设$I_{ij},i<j$为指示器变量,$I_{ij}=\left\{\begin{array}{cl}
	1,&A[i]>A[j]\\
	0,&\text{否则}
	\end{array}\right.$。设逆序对的个数为$X$,则:
	\[
	\begin{aligned}
		E[X]&=E\left[\sum_{i<j}I_{ij}\right]\\
		&=\sum_{i=1}^{n-1}\sum_{j=i+1}^{n}I_{ij}\\
		&=\sum_{i=1}^{n-1}\sum_{j=i+1}^{n}P\left(A_i>A_j\right)\\
		&=\frac{1}{2}\sum_{i=1}^{n-1}n-i\\
		&=\frac{1}{2}\sum_{i=1}^{n-1}i\\
		&=\frac{n(n-1)}{4}
	\end{aligned}
	\]
	\subsection*{5.3.2}
	不能,考虑若$A=\left\{1,2,3\right\}$,则此算法不能生成排列$\left\{3,2,1\right\}$。
	\subsection*{5.2.3}
	产生各种排列的概率并不相等,考虑$n=3$时,循环进行了$3$次,在三个位置选择元素放入,则共有$3^3=27$种方式,它们出现的概率相等。但$3$个元素的全排列共有$3!=6$种,不是$27$的因数,所以这$27$种选择方案不能均等映射到$6$种全排列上。故得到各种全排列的概率不等。
	\subsection*{5.2.4}
	\begin{proof}
		由\id{offset}等可能的取$1-n$中的所有值,所以$A$中任意元素$A[i]$落在$B[i+1],B[i+2],\dots,B[i]$的概率均为$1/n$。\par 
		下证此算法得到的全排列概率并不均等:\par 
		考虑元素$A[i],A[i+1],1\le i\le n-1$,则由于这是两个任意的元素,在$B$中,$A[i]$在$A[i+1]$之前或之后的概率相等。而使用此算法得到的排列中,有$1-\frac{i+1}{n}$的概率$A[i]$仍在$A[i+1]$前,$\frac{i+1}{n}$的概率$A[i]$在$A[j]$后,与均匀的全排列不符,故此算法得到的排列并不是均匀分布的。
	\end{proof}
\end{document}