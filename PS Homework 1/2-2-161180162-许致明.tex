\documentclass[twocolumn]{article}
\usepackage{verbatim}
\usepackage{amsfonts}
\usepackage{geometry}
\usepackage{amsmath}
\usepackage{amsthm}
\usepackage{amssymb}
\usepackage{listings}
\usepackage{graphicx}
\usepackage{clrscode3e}
\usepackage{txfonts}
\usepackage{enumerate}
\usepackage{ctex}
\usepackage{txfonts}
\usepackage{fontspec-xetex}
\geometry{top=2.5cm,bottom=2.5cm,left=2.5cm,right=2.5cm}
\setlength\parindent{0em}
\setmainfont{Times New Roman}
\begin{document}
	\title{问题求解(二)作业(第二周)}\author{161180162 许致明}\maketitle
	\section*{DH第六章}
	\subsection*{6.1}
	\begin{enumerate}[(a)]
		\item 将此过程改为如下形式:
		\begin{codebox}
			\zi 
			\proc{Salary-Computation$(\id{N},\id{BT[1..N]})$}\li
			\For $\id{I}\gets 1$ \To $\id{N}$\Do\li 
			\If $\id{BT[I]}\le \id{M}$\Do\li 
			$\id{tmp}\gets \id{BT[I]}\times \id{Rl}$\li 
			$\id{AT[I]}\gets \id{BT[I]}-\id{tmp}$\li 
			$\id{Tl}\gets \id{Tl}+\id{tmp}$\li 
			\Else\li 
			$\id{tmp}\gets \id{BT[I]}\times \id{Rh}$\li 
			$\id{AT[I]}\gets \id{BT[I]}-\id{tmp}$\li 
			$\id{Th}\gets \id{Th}+\id{tmp}$\End  
		\end{codebox}
	共减少了$N$次比较和$2N$次乘法,时间复杂度大约下降了一半。
	\item 在这种情况下,可以只使用一个数组$\id{B[N]}$来储存税后的工资信息,这种做法在$\id{N}$较大时,可以使空间复杂度下降到原来的一半。
	\end{enumerate}
	\subsection*{6.8}
	\begin{enumerate}[(a)]
		\item 此算法进行了一次循环,即将指针从需搜索的文本头移动到文本尾,因此为线性复杂度。
		\item 此算法在进行旋转时,至多贴合多边形的每一条边一次,因此为线性复杂度。
	\end{enumerate}
	\subsection*{6.10}
	设所有节点的个数为$n$:\\
	4.2(a)(b)(c)中最坏情况的复杂度均为$O(n)$,即需要访问所有节点以获得所需信息\\
	4.3(a)对于一个合理的较小的常数$K$,打印第$K$层的的节点数字和,则复杂度为$O(\log_2 N)$。4.3(b)最坏情况下需要遍历整棵树,故复杂度为$O(n)$。
	\subsection*{6.13}
	\begin{proof}
		基于比较的排序可以被抽象为\emph{决策树},它是一颗满二叉树,表示排序算法作用于给定输入的所有比较。考虑一棵高度为$h$,具有$l$个可达叶节点的决策树,它对应于$n$个元素所做的比较排序。因为$n$个元素的输入共有$n!$种排序,每一种都作为一个叶子出现在树中,所以有$n!\le l$。又因为在一棵高度为$h$的二叉树中,叶子的数目不多于$2^h$,则有:
		\[
		n!\le l\le 2^h
		\]
		取对数有:
		\[
		\begin{aligned}
			h&\ge \log_2(n!)\\
			&=O(n\log_2 n)
		\end{aligned}
		\]
		因此,最少的比较次数(对应于树的高度)为$O(n\log_2 n)$
	\end{proof}
	\subsection*{6.18}
	\begin{codebox}
		\zi \proc{LG1$(\id{m},\id{n})$}\Comment{一种比较naive的方法} \li
		$\id{ret}=m$\li 
		\While ($\id{n}>\id{ret}$)\Do\li 
		$\id{ret}\gets \id{ret}\times \id{m}$\End\li 
		\Return $\id{ret}$
	\end{codebox}
\end{document}